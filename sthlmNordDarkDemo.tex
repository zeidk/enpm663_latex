%=+=+=+=+=+=+=+=+=+=+=+=+=+=+=+=+=+=+=+=+=+=+=+=+=+=+=+=+=+=+=+=+=+=+=+=+=+=+=+=
%             _             _
%            | |           | |
%  _ __ ___  | |__    ___  | | ___   ___   _ __       ___   ___   _ __ ___ 
% | '_ ` _ \ | '_ \  / _ \ | |/ __| / _ \ | '_ \     / __| / _ \ | '_ ` _ \ 
% | | | | | || | | || (_) || |\__ \| (_) || | | | _ | (__ | (_) || | | | | | 
% |_| |_| |_||_| |_| \___/ |_||___/ \___/ |_| |_|(_) \___| \___/ |_| |_| |_| 
%
% Author: Mark H. Olson
% Website: https://mholson.com
% Github: https://github.com/mholson
%
% Updated: 2022-07-08 > Version 3.1415
% - porting for ease of use with overleaf.com
% - changed code printing behavior by adding default support of listings pkg
% while leaving support for custom minted pkg using codeminted documentclass
% option
% Updated: 2022-04-28 > Version 3.141
% - fixed lists by removing enumitem package
% Updated: 2022-04-13 > Version 3.14
% - added optional bibliography
% - added template file
% Updated: 2022-04-13 Version 3.1
% Created: 2015-07-31 Version 3.0
%
%=+=+=+=+=+=+=+=+=+=+=+=+=+=+=+=+=+=+=+=+=+=+=+=+=+=+=+=+=+=+=+=+=+=+=+=+=+=+=+=

%=-=-=-=-=-=-=-=-=-=-=-=-=-=-=-=-=-=-=-=-=-=-=-=-=-=-=-=-=-=-=-=-=-=-=-=-=-=-=-=
% PREAMBLE :: sthlmNordLightDemo.tex 
%=-=-=-=-=-=-=-=-=-=-=-=-=-=-=-=-=-=-=-=-=-=-=-=-=-=-=-=-=-=-=-=-=-=-=-=-=-=-=-=
%
% > > >	The following beamer class options are available
%		aspectratio=169		Change aspect ratio to 16:9	
%		bibref				Include bibliography
%		sectionpages		Show section pages
%		codeminted			use minted pkg for code printing instead of listings
%							(requires additional setup & Python installed)
% 		font sizes 			{8, 9, 10, 11, 12, 14, 17, 20} 11 Default
%
% > > > The following sthlmnord package options are available
%		mode				= dark (default)
%							= light
%=-=-=-=-=-=-=-=-=-=-=-=-=-=-=-=-=-=-=-=-=-=-=-=-=-=-=-=-=-=-=-=-=-=-=-=-=-=-=-=

\documentclass[aspectratio=169, sectionpages, bibref]{beamer}
% > > > Bibliography File 
\newcommand{\bibfilename}{mhoreferences.bib}
% > > > Choose Theme
%\usetheme[mode=light]{sthlmnord}
\usetheme{sthlmnord}

% > > > Generate some Lorem Ipsum placeholder text for the demo.
\usepackage{lipsum}

% > > >	Image File Paths
% 		Here you can add one or more paths to where your images are being
%		stored.  This will allow you to include only the image file 
%		name when placing it into your document.
%\graphicspath{{path1},{path2},{path3}}
\graphicspath{{./assets/}} 
% > > >	Optional use of using subfiles to make content more modular
\usepackage{subfiles}

% > > > Document Information
\title{sthlmNord Beamer Theme [version round (pi, 4)]}
\subtitle{Nord Inspired by Stockholm}
\newcommand{\titleAuthor}{Created by}
\author{mholson.com}
\newcommand{\titleInstitute}{Institute}
\institute{School in Stockholm}
\newcommand{\titleMiscI}{Course}
\newcommand{\descMiscI}{Courses Title Goes Here}
\newcommand{\titleMiscII}{File}
\newcommand{\descMiscII}{\currfilebase}
\date{\today}
\titlegraphic{nordtitlelogolight}

% > > > pdf customizations via hyperref (pkg installed by beamer)
\hypersetup{
%colorlinks=true,
% You might want to disable color links for you final draft 
% AND for colors to work properly where links are involved.
colorlinks=false,
linkcolor={nordNine},
citecolor={nordNine},
urlcolor={nordNine}
}

%=-=-=-=-=-=-=-=-=-=-=-=-=-=-=-=-=-=-=-=-=-=-=-=-=-=-=-=-=-=-=-=-=-=-=-=-=-=-=-=
%
%    DOCUMENT BEGINS HERE 
%
%=-=-=-=-=-=-=-=-=-=-=-=-=-=-=-=-=-=-=-=-=-=-=-=-=-=-=-=-=-=-=-=-=-=-=-=-=-=-=-=
\begin{document}

%=-=-=-=-=-=-=-=-=-=-=-=-=-=-=-=-=-=-=-=-=-=-=-=-=-=-=-=-=-=-=-=-=-=-=-=-=-=-=-=
%   TITLE START   -=-=-=-=-=-=-=-=-=-=-=-=-=-=-=-=-=-=-=-=-=-=-=-=-=-=-=-=-=-=-=
\titlepage%
\subfile{0-slides/tex.slide.sthmlNordCover}
%   TITLE END   --==-=-=-=-=-=-=-=-=-=-=-=-=-=-=-=-=-=-=-=-=-=-=-=-=-=-=-=-=-=-=
%=-=-=-=-=-=-=-=-=-=-=-=-=-=-=-=-=-=-=-=-=-=-=-=-=-=-=-=-=-=-=-=-=-=-=-=-=-=-=-=

%=-=-=-=-=-=-=-=-=-=-=-=-=-=-=-=-=-=-=-=-=-=-=-=-=-=-=-=-=-=-=-=-=-=-=-=-=-=-=-=
%   TABLE OF CONTENTS START   -=-=-=-=-=-=-=-=-=-=-=-=-=-=-=-=-=-=-=-=-=-=-=-=-=
\begin{frame}
	\frametitle{Table of contents}
	% > > > For longer presentations use \tableofcontents[hideallsubsections] option
	%		It is also possible to manually control the entries of the table of 
	% 		contents by sections.
	%\tableofcontents[sections={1-10}]
	%\framebreak
	%\tableofcontents[sections={11-15}]
	\tableofcontents
\end{frame}
%   TABLE OF CONTENTS END   -=-=-=-=-=-=-=-=-=-=-=-=-=-=-=-=-=-=-=-=-=-=-=-=-=-=
%=-=-=-=-=-=-=-=-=-=-=-=-=-=-=-=-=-=-=-=-=-=-=-=-=-=-=-=-=-=-=-=-=-=-=-=-=-=-=-=

%=-=-=-=-=-=-=-=-=-=-=-=-=-=-=-=-=-=-=-=-=-=-=-=-=-=-=-=-=-=-=-=-=-=-=-=-=-=-=-=
% SECTION 
%=-=-=-=-=-=-=-=-=-=-=-=-=-=-=-=-=-=-=-=-=-=-=-=-=-=-=-=-=-=-=-=-=-=-=-=-=-=-=-=
\section{Background Information}

\subfile{0-slides/tex.slide.useMetroWarning}
\subfile{0-slides/tex.slide.majorFeatures}
\subfile{0-slides/tex.slide.history}
\subfile{0-slides/tex.slide.noGuarantee}
\subfile{0-slides/tex.slide.availableGitHub}
\subfile{0-slides/tex.slide.availableOverleaf}
\subfile{0-slides/tex.slide.ctanPackages}
\subfile{0-slides/tex.slide.customPackages}

%=-=-=-=-=-=-=-=-=-=-=-=-=-=-=-=-=-=-=-=-=-=-=-=-=-=-=-=-=-=-=-=-=-=-=-=-=-=-=-=
% SECTION 
%=-=-=-=-=-=-=-=-=-=-=-=-=-=-=-=-=-=-=-=-=-=-=-=-=-=-=-=-=-=-=-=-=-=-=-=-=-=-=-=
\section{Colors}

\subfile{0-slides/tex.slide.nordPaletteDark}
\subfile{0-slides/tex.slide.nordPaletteTextDark}
\subfile{0-slides/tex.slide.nordPaletteTextHighlightDark}

%=-=-=-=-=-=-=-=-=-=-=-=-=-=-=-=-=-=-=-=-=-=-=-=-=-=-=-=-=-=-=-=-=-=-=-=-=-=-=-=
% SECTION 
%=-=-=-=-=-=-=-=-=-=-=-=-=-=-=-=-=-=-=-=-=-=-=-=-=-=-=-=-=-=-=-=-=-=-=-=-=-=-=-=
\section{Deck Structures}

\subfile{0-slides/tex.slide.blocks}
\subfile{0-slides/tex.slide.listsEnumerate}
\subfile{0-slides/tex.slide.listsItemize}
\subfile{0-slides/tex.slide.listsDescription}
\subfile{0-slides/tex.slide.codeblocks}

\subfile{0-slides/tex.slide.exampleSlide}
\subfile{0-slides/tex.slide.theoremSlide}

%=-=-=-=-=-=-=-=-=-=-=-=-=-=-=-=-=-=-=-=-=-=-=-=-=-=-=-=-=-=-=-=-=-=-=-=-=-=-=-=
% SECTION 
%=-=-=-=-=-=-=-=-=-=-=-=-=-=-=-=-=-=-=-=-=-=-=-=-=-=-=-=-=-=-=-=-=-=-=-=-=-=-=-=
\section{Fonts}

\subfile{0-slides/tex.slide.fonts}

%=-=-=-=-=-=-=-=-=-=-=-=-=-=-=-=-=-=-=-=-=-=-=-=-=-=-=-=-=-=-=-=-=-=-=-=-=-=-=-=
% SECTION 
%=-=-=-=-=-=-=-=-=-=-=-=-=-=-=-=-=-=-=-=-=-=-=-=-=-=-=-=-=-=-=-=-=-=-=-=-=-=-=-=
\section{Mathematics}

\subfile{0-slides/tex.slide.typeMathematics}
\subfile{0-slides/tex.slide.tikz}
\subfile{0-slides/tex.slide.mathExampleExpand}

\subfile{0-slides/tex.slide.mathExampleCTS}
\subfile{0-slides/tex.slide.mathExampleDiceCoins.tex}
\subfile{0-slides/tex.slide.mathSets}

%=-=-=-=-=-=-=-=-=-=-=-=-=-=-=-=-=-=-=-=-=-=-=-=-=-=-=-=-=-=-=-=-=-=-=-=-=-=-=-=
% SECTION 
%=-=-=-=-=-=-=-=-=-=-=-=-=-=-=-=-=-=-=-=-=-=-=-=-=-=-=-=-=-=-=-=-=-=-=-=-=-=-=-=
\section{References}

\begin{frame}[allowframebreaks]{References}
	\printbibliography[title={References}]%
\end{frame}

\end{document}
%=+=+=+=+=+=+=+=+=+=+=+=+=+=+=+=+=+=+=+=+=+=+=+=+=+=+=+=+=+=+=+=+=+=+=+=+=+=+=+=
% END OF FILE
%=+=+=+=+=+=+=+=+=+=+=+=+=+=+=+=+=+=+=+=+=+=+=+=+=+=+=+=+=+=+=+=+=+=+=+=+=+=+=+=

